\documentclass{report}
\usepackage[utf8]{inputenc}
\usepackage[russian]{babel}
\usepackage{setspace,amsmath}
\usepackage{amssymb}
\usepackage{amsthm}
\usepackage{amsfonts}
\usepackage{scalerel}
\usepackage{graphicx}
\usepackage{float}
\usepackage{wrapfig}
\usepackage[unicode, pdftex]{hyperref}

\def\stretchint#1{\vcenter{\hbox{\stretchto[440]{\displaystyle\int}{#1}}}}
\def\scaleint#1{\vcenter{\hbox{\scaleto[3ex]{\displaystyle\int}{#1}}}}

\theoremstyle{definition}
\newtheorem{definition}{Определение}[section]
\newtheorem{example}{Пример}
\newtheorem*{effect}{Следствие}
\newtheorem{statement}{Утверждение}[section]
\newtheorem*{remark}{Замечание}
\newtheorem{lemma}{Лемма}[section]
\newtheorem{theorem}{Теорема}[section]

\title{Дифференциальные уравнения \\ 3 семестр}
\author{Данил Заблоцкий}
\date{\today}

\begin{document}

\maketitle
\tableofcontents

\chapter{Основные понятия}

\section{Уравнение $1$-го порядка}

\begin{definition}[Дифференциальное уравнение $n$-го порядка]
    \textbf{Дифференциальным уравнением $n$-го порядка} называется уравнение вида
    \begin{equation}
        F(x,y,y',\ldots,y^{(n)}) = 0
    \end{equation}
    \begin{equation*}
        x \in (a,b) \subset \mathbb{R}, \quad -\infty \leqslant a < b \leqslant +\infty
    \end{equation*}
    \begin{equation*}
        [a,b), \quad [a,b], \quad (a,b]
    \end{equation*}
\end{definition}

\begin{definition}[Дифференциальное уравнение, разрешенное относительно старшей производной]
    \textbf{Дифференциальным уравнением, разрешенным относительно старшей производной} называется уравнение вида
    \begin{equation}
        y^{(n)} = f(x,y,y',\ldots,y^{(n-1)}), \quad x \in (a,b)
    \end{equation}
\end{definition}

\begin{definition}[Решение дифференциального уравнения]
    \textbf{Решением дифференциального уравнения} $(1.1)$ или $(1.2)$ называется $n$ раз дифференцируемая функция $y = \phi(x)$ на интервале $(a,b)$, если при подстановке она обращает уравнение в тождество на этом интервале.
\end{definition}

\begin{remark}
    \begin{equation*}
        y = \frac{1}{x+1}, \quad (-\infty, -1) \cup (-1, -\infty)
    \end{equation*}

    Предмет дифференциального уравнения:
    \begin{enumerate}
        \item Решение дифференциального уравнения.
        \item Существует ли решение на $(a,b)$?
        \item Единственность, $y(x_0)=y_0$ (задача Коши).
        \item О продолжении.
        \item Свойства решения: \begin{itemize}
                  \item ограниченность
                  \item монотонность
                  \item поведение решения вблизи границ ($x \rightarrow +\infty$)
                  \item нули функции на $(a,b)$
              \end{itemize}
    \end{enumerate}
\end{remark}

\begin{definition}[Дифференциальное уравнение $1$-го порядка]
    \textbf{Дифференциальным уравнением $1$-го порядка} называется уравнение вида
    \begin{equation}
        F(x,y,y')=0, \quad x \in (a,b)
    \end{equation}
    \begin{center}
        (неразрешенное относительно $y'$)
    \end{center}
\end{definition}

\begin{definition}[Дифференциальное уравнение, разрешеноое относительно первой производной]
    \textbf{Дифференциальным уравнением $1$-го порядка, разрешенным относительно первой производной}, называется уравнение вида
    \begin{equation}
        y'=f(x,y), \quad x \in (a,b)
    \end{equation}
\end{definition}

\begin{definition}[Решение дифференциального уравнения $(1.3)$ и $(1.4)$]
    \textbf{Решением дифференциального уравнения} $(1.3)$ и $(1.4)$ называется дифференцируемая функция $y = \phi(x)$, обращающая уравнение в тождество на этом интервале.
\end{definition}

\begin{example}
    $y' = - \frac{x}{y}$ имеет решение $x^2 + y^2 = c$, где $c$ - произвольная константа, $c > 0$.
\end{example}

\begin{definition}[Поле направлений]
    Сопоставим любой точке $(x_0,y_0) \rightarrow y'(x_0) = f(x_0,y_0) = \tan\alpha$ направления $l$. Семейство (совокупность) направлений $l$ дает \textbf{поле направлений}.
\end{definition}

\begin{definition}[Интегральная кривая]
    Кривая, касающаяся в каждой своей точке поля направлений, называется \textbf{интегральной кривой}.
    \begin{center}
        $y=\phi(x,c) \quad$ интегральная кривая $\equiv$ график решений
    \end{center}
\end{definition}

\begin{definition}[Изоклины]
    Кривые, вдоль которых поле направлений постоянно, называется \textbf{изоклинами}.
\end{definition}

\begin{example}
    $y' = y-x^2 \quad$ Напишем уравнение изоклин: $y-x^2 = c$ (заменяем $y'$ на $c$)
    \begin{enumerate}
        \item $c=0 \implies y-x^2 = 0 \implies y=x^2$

              $\tan \alpha = 0 \implies \alpha = 0; \quad y \ const$.
        \item $c=1 \implies y-x^2 = 1 \implies y=x^2 + 1$

              $\tan \alpha = 1 \implies \alpha = 45^{\circ}; \quad y\uparrow$
        \item $c=2 \implies y-x^2 = 2 \implies y=x^2 + 2$

              $\tan \alpha = 2 \implies \alpha = \arctan 2; \quad y\uparrow$
        \item $c=-1 \implies y-x^2 = -1 \implies y=x^2 - 1$

              $\tan \alpha = -1 \implies \alpha = -45^{\circ}; \quad y\downarrow$
        \item $c=-2 \implies y-x^2 = -2 \implies y=x^2 - 2$

              $\tan \alpha = -2 \implies \alpha = -\arctan 2; \quad y\downarrow$
    \end{enumerate}
    \begin{equation*}
        y' = 0
    \end{equation*}
    \begin{equation*}
        y' > 0, \quad y > x^2
    \end{equation*}
    \begin{equation*}
        y' < 0, \quad y < x^2
    \end{equation*}
\end{example}

\begin{definition}[Общее решение]
    \textbf{Общее решение} - совокупность функций, которая содержит все решения уравнения.

    Если решение задается функцией $y = \phi(x,c)$ или $\psi(x,y,c) = 0$, то общее решение должно удовлетворять условиям:
    \begin{enumerate}
        \item При любом $c$ формула дает решение уравнение.
        \item Любое решение уравнения находится по формуле при некотором $c = c_0$.
    \end{enumerate}
\end{definition}

\begin{definition}[Частное решение]
    \textbf{Частное решение} определяется из общего при некотором $c = c_0$.
\end{definition}

\begin{example}
    $y'=x \implies y = \frac{x^2}{2}+c$ - общее решение

    при $c=0: \ y = \frac{x^2}{2}$, при $c=1: \ y=\frac{x^2}{2} + 1$ - частное решение
\end{example}

\section{Уравнения с разделяющимися переменными}

\begin{definition}[Уравнения с разделяющимися переменными]
    \textbf{Уравнениями с разделяющимися переменными} называются уравнения вида:
    \begin{equation*}
        y'=f(x)\cdot g(y), \quad f_1(x) \cdot g_1(y)\cdot dx + f_2(x) \cdot g_2(y)\cdot dy = 0,
    \end{equation*}
    $f, \ f_1, \ f_2$ зависят от $x$, $g, \ g_1, \ g_2$ зависят от $y$. \\

    Алгоритм: \\

    $\left[
        \begin{array}{l}
            g(y) = 0 \implies y = c \\
            \left\{
            \begin{array}{l}
                g(y) \ne 0 \\
                \frac{y'}{g(y)} = f(x)
            \end{array}
            \right. \implies \int \frac{y'dx}{g(y)} = \int f(x)dx \implies \int \frac{dy}{g(y)} = \int f(x)dx \implies\end{array}
        \right.$ \\

    $\left[ \begin{array}{l}
            y = \phi(x,c) \\
            \psi (x,y,c) = 0
        \end{array}\right. \iff \left[\begin{array}{l}
            y = c_1 \\
            \left[\begin{array}{l}
                      y = \phi(y,c_2) \\
                      \psi(x,y,c_2) = 0
                  \end{array}
            \right.
        \end{array}\right.$
\end{definition}

\begin{example}
    $y' = xy^2$ \\

    $\left[\begin{array}{l}
            y = 0 \\
            \left\{\begin{array}{l}
                       \frac{dy}{y^2} = xdx \\
                       y \ne 0
                   \end{array}\right.
        \end{array}\right. \iff \int \frac{dy}{y^2} = \int xdx \implies -\frac{1}{y} = \frac{x^2}{2} + C$ \\

    $\left[\begin{array}{l}
            y = -\frac{2}{x^2 + 2C}, \ C \in \mathbb{R} \\
            y = 0
        \end{array}\right.$
\end{example}

\begin{theorem}[Задача Коши]
    $\left\{\begin{array}{l}
            y'=f(x,y) \\
            y(x_0) = y_0
        \end{array}\right.$ \\

    $f(x,y) \in C(D), \quad (x_0, y_0) \in D$ (РИСУНКИ)
\end{theorem}

\begin{example}
    $y' = \sqrt{y}$ \\

    $\left[\begin{array}{l}
            y = 0 \\
            \left\{\begin{array}{l}
                       \frac{dy}{\sqrt{y}} = \int dx \\
                       y \ne 0
                   \end{array}\right.
        \end{array}\right. \iff 2\sqrt{y} = x + C \implies y = (\frac{x + c}{2})^2$ при $x + c \geqslant 0$. \\

    \begin{enumerate}
        \item $y = 0 \ \cup$ парабола $AB_1D_1$;
        \item $x_0$ на кривой $y = 0 \left[\begin{array}{l}
                      y = 0   \\
                      ABD     \\
                      AB_1D_1 \\
                      AB_2D_2
                  \end{array}\right.$
    \end{enumerate}

    Ответ: $\left[\begin{array}{l}
            y = 0 \\
            y = (\frac{x + c}{2})^2, \quad x + c \geqslant 0
        \end{array}\right.$
\end{example}

\begin{definition}[Точка единственности решения]
    Точка $(x_0, y_0)$ называется \textbf{точкой единственности решения} $y = \phi(x)$, если через нее не проходит другое решение, не совпадающее с решением $y = \phi(x)$ ни в какой окрестности этой точки.

    Остальные точки называются \textbf{точками неединственности}.

    Решение, которое содержит точки неединственности, называется \textbf{особым решением}.
\end{definition}

\begin{theorem}[$\exists$ и $!$-ть решения задачи Коши]
    Пусть
    \begin{center}
        $f(x,y)$ в $\left\{\begin{array}{l}
                y'=f(x,y) \\
                y(x_0) = y_0
            \end{array}\right.$
    \end{center}

    \begin{enumerate}
        \item Определена и непрерывна в прямоугольнике в прямоугольнике \\
              $\Pi = \{(x,y): \ |x - x_0| \leqslant a, \ |y - y_0| \leqslant b\}$
        \item Удовлетворяет условию Липшица по $y$ в $\Pi$ ($f_y'(x,y)$ непрерывна в $\Pi$)
    \end{enumerate}

    Тогда $\exists !$ решение задачи $\left\{\begin{array}{l}
            y'=f(x,y) \\
            y(x_0) = y_0
        \end{array}\right.$ в окрестности точки \\
    $x_0 \ (x_0 - h, \ x_0 + h)$, где $h = \min(a, \frac{b}{M}), \ M = \max|f(x,y)|, \ (x,y) \in \Pi$. (РИСУНОК)
\end{theorem}

\begin{definition}
    $f(x,y)$ удовлетворяет условию Липшица по переменной $y$, если $\exists L > 0$ такая, что $\forall (x,y_1)$ и $(x,y_2)$ имеет место $|f(x,y_1) - f(x,y_2)| \leqslant L \cdot|y_1 - y_2|$.

    Если $f_y'(x,y)$ - непрерывна в $\Pi$, то выполняется условие Липшица.

    $\forall (x,y_1), \ (x,y_2) \in \Pi, \ \exists \widetilde{y} \in [y_1,y_2]$.

    $|f(x,y_1) - f(x,y_2)| \leqslant |f_y'(x,\widetilde{y}) \cdot (y_1 - y_2)| \leqslant |f_y'(x,\widetilde{y})||y_1 - y_2| = L|y_1 - y_2|$.
\end{definition}

\begin{example}
    $y' = \frac{1}{y^2}, \ f(x,y) = \frac{1}{y^2}, \ f_y'=\frac{2}{y^3}, \quad \int y^2dy = \int ydx \implies \frac{y^3}{3} = x + c \implies
        \left\{\begin{array}{l}
            y = \sqrt[3]{3(x + c)} \\
            y(x_0) = y_0
        \end{array}\right. \implies y = \sqrt[3]{3(x-x_0) + y_0^3}$
\end{example}

\begin{example}
    $y'=sign x = \left\{\begin{array}{rl}
            1,  & x > 0 \\
            0,  & x = 0 \\
            -1, & x < 0
        \end{array}\right.$
\end{example}

\begin{example}
    $y' = y^2 - 2y + 1 = (y-1)^2$

    $\left[\begin{array}{l}
            y = 1 \\
            \left\{\begin{array}{l}
                       y \ne 1 \\
                       \frac{dy}{(y - 1)^2} = \int dx
                   \end{array}\right.
        \end{array}\right. \iff \frac{1}{y-1} = x + C \implies y = 1 - \frac{1}{x + C}$
\end{example}

\end{document}