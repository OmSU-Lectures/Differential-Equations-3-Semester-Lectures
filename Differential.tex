\documentclass{report}
\usepackage[utf8]{inputenc}
\usepackage[russian]{babel}
\usepackage{setspace,amsmath}
\usepackage{amssymb}
\usepackage{amsthm}
\usepackage{amsfonts}
\usepackage{scalerel}
\usepackage{graphicx}
\usepackage{float}
\usepackage{wrapfig}
\usepackage[unicode, pdftex]{hyperref}

\def\stretchint#1{\vcenter{\hbox{\stretchto[440]{\displaystyle\int}{#1}}}}
\def\scaleint#1{\vcenter{\hbox{\scaleto[3ex]{\displaystyle\int}{#1}}}}

\theoremstyle{definition}
\newtheorem{definition}{Определение}[section]
\newtheorem{example}{Пример}
\newtheorem*{effect}{Следствие}
\newtheorem{statement}{Утверждение}[section]
\newtheorem*{remark}{Замечание}
\newtheorem{lemma}{Лемма}[section]
\newtheorem{theorem}{Теорема}[section]

\newcommand{\RomanNumeralCaps}[1]{\MakeUppercase{\romannumeral #1}}

\title{Дифференциальные уравнения \\ 3 семестр}
\author{Данил Заблоцкий}
\date{\today}

\begin{document}

\maketitle
\tableofcontents

\chapter{Основные понятия}

\section{Уравнение $1$-го порядка}

\begin{definition}[Дифференциальное уравнение $n$-го порядка]
    \textbf{Дифференциальным уравнением $n$-го порядка} называется уравнение вида
    \begin{equation}
        F(x,y,y',\ldots,y^{(n)}) = 0
    \end{equation}
    \begin{equation*}
        x \in (a,b) \subset \mathbb{R}, \quad -\infty \leqslant a < b \leqslant +\infty
    \end{equation*}
    \begin{equation*}
        [a,b), \quad [a,b], \quad (a,b]
    \end{equation*}
\end{definition}

\begin{definition}[Дифференциальное уравнение, разрешенное относительно старшей производной]
    \textbf{Дифференциальным уравнением, разрешенным относительно старшей производной} называется уравнение вида
    \begin{equation}
        y^{(n)} = f(x,y,y',\ldots,y^{(n-1)}), \quad x \in (a,b)
    \end{equation}
\end{definition}

\begin{definition}[Решение дифференциального уравнения]
    \textbf{Решением дифференциального уравнения} $(1.1)$ или $(1.2)$ называется $n$ раз дифференцируемая функция $y = \phi(x)$ на интервале $(a,b)$, если при подстановке она обращает уравнение в тождество на этом интервале.
\end{definition}

\begin{remark}
    \begin{equation*}
        y = \frac{1}{x+1}, \quad (-\infty, -1) \cup (-1, -\infty)
    \end{equation*}

    Предмет дифференциального уравнения:
    \begin{enumerate}
        \item Решение дифференциального уравнения.
        \item Существует ли решение на $(a,b)$?
        \item Единственность, $y(x_0)=y_0$ (задача Коши).
        \item О продолжении.
        \item Свойства решения: \begin{itemize}
                  \item ограниченность
                  \item монотонность
                  \item поведение решения вблизи границ ($x \rightarrow +\infty$)
                  \item нули функции на $(a,b)$
              \end{itemize}
    \end{enumerate}
\end{remark}

\begin{definition}[Дифференциальное уравнение $1$-го порядка]
    \textbf{Дифференциальным уравнением $1$-го порядка} называется уравнение вида
    \begin{equation}
        F(x,y,y')=0, \quad x \in (a,b)
    \end{equation}
    \begin{center}
        (неразрешенное относительно $y'$)
    \end{center}
\end{definition}

\begin{definition}[Дифференциальное уравнение, разрешеноое относительно первой производной]
    \textbf{Дифференциальным уравнением $1$-го порядка, разрешенным относительно первой производной}, называется уравнение вида
    \begin{equation}
        y'=f(x,y), \quad x \in (a,b)
    \end{equation}
\end{definition}

\begin{definition}[Решение дифференциального уравнения $(1.3)$ и $(1.4)$]
    \textbf{Решением дифференциального уравнения} $(1.3)$ и $(1.4)$ называется дифференцируемая функция $y = \phi(x)$, обращающая уравнение в тождество на этом интервале.
\end{definition}

\begin{example}
    $y' = - \frac{x}{y}$ имеет решение $x^2 + y^2 = c$, где $c$ - произвольная константа, $c > 0$.
\end{example}

\begin{definition}[Поле направлений]
    Сопоставим любой точке $(x_0,y_0) \rightarrow y'(x_0) = f(x_0,y_0) = \tan\alpha$ направления $l$. Семейство (совокупность) направлений $l$ дает \textbf{поле направлений}.
\end{definition}

\begin{definition}[Интегральная кривая]
    Кривая, касающаяся в каждой своей точке поля направлений, называется \textbf{интегральной кривой}.
    \begin{center}
        $y=\phi(x,c) \quad$ интегральная кривая $\equiv$ график решений
    \end{center}
\end{definition}

\begin{definition}[Изоклины]
    Кривые, вдоль которых поле направлений постоянно, называется \textbf{изоклинами}.
\end{definition}

\begin{example}
    $y' = y-x^2 \quad$ Напишем уравнение изоклин: $y-x^2 = c$ (заменяем $y'$ на $c$)
    \begin{enumerate}
        \item $c=0 \implies y-x^2 = 0 \implies y=x^2$

              $\tan \alpha = 0 \implies \alpha = 0; \quad y \ const$.
        \item $c=1 \implies y-x^2 = 1 \implies y=x^2 + 1$

              $\tan \alpha = 1 \implies \alpha = 45^{\circ}; \quad y\uparrow$
        \item $c=2 \implies y-x^2 = 2 \implies y=x^2 + 2$

              $\tan \alpha = 2 \implies \alpha = \arctan 2; \quad y\uparrow$
        \item $c=-1 \implies y-x^2 = -1 \implies y=x^2 - 1$

              $\tan \alpha = -1 \implies \alpha = -45^{\circ}; \quad y\downarrow$
        \item $c=-2 \implies y-x^2 = -2 \implies y=x^2 - 2$

              $\tan \alpha = -2 \implies \alpha = -\arctan 2; \quad y\downarrow$
    \end{enumerate}
    \begin{equation*}
        y' = 0
    \end{equation*}
    \begin{equation*}
        y' > 0, \quad y > x^2
    \end{equation*}
    \begin{equation*}
        y' < 0, \quad y < x^2
    \end{equation*}
\end{example}

\begin{definition}[Общее решение]
    \textbf{Общее решение} - совокупность функций, которая содержит все решения уравнения.

    Если решение задается функцией $y = \phi(x,c)$ или $\psi(x,y,c) = 0$, то общее решение должно удовлетворять условиям:
    \begin{enumerate}
        \item При любом $c$ формула дает решение уравнение.
        \item Любое решение уравнения находится по формуле при некотором $c = c_0$.
    \end{enumerate}
\end{definition}

\begin{definition}[Частное решение]
    \textbf{Частное решение} определяется из общего при некотором $c = c_0$.
\end{definition}

\begin{example}
    $y'=x \implies y = \frac{x^2}{2}+c$ - общее решение

    при $c=0: \ y = \frac{x^2}{2}$, при $c=1: \ y=\frac{x^2}{2} + 1$ - частное решение
\end{example}

\section{Уравнения с разделяющимися переменными}

\begin{definition}[Уравнения с разделяющимися переменными]
    \textbf{Уравнениями с разделяющимися переменными} называются уравнения вида:
    \begin{equation*}
        y'=f(x)\cdot g(y), \quad f_1(x) \cdot g_1(y)\cdot dx + f_2(x) \cdot g_2(y)\cdot dy = 0,
    \end{equation*}
    $f, \ f_1, \ f_2$ зависят от $x$, $g, \ g_1, \ g_2$ зависят от $y$. \\

    Алгоритм: \\

    $\left[
        \begin{array}{l}
            g(y) = 0 \implies y = c \\
            \left\{
            \begin{array}{l}
                g(y) \ne 0 \\
                \frac{y'}{g(y)} = f(x)
            \end{array}
            \right. \implies \int \frac{y'dx}{g(y)} = \int f(x)dx \implies \int \frac{dy}{g(y)} = \int f(x)dx \implies\end{array}
        \right.$ \\

    $\left[ \begin{array}{l}
            y = \phi(x,c) \\
            \psi (x,y,c) = 0
        \end{array}\right. \iff \left[\begin{array}{l}
            y = c_1 \\
            \left[\begin{array}{l}
                      y = \phi(y,c_2) \\
                      \psi(x,y,c_2) = 0
                  \end{array}
            \right.
        \end{array}\right.$
\end{definition}

\begin{example}
    $y' = xy^2$ \\

    $\left[\begin{array}{l}
            y = 0 \\
            \left\{\begin{array}{l}
                       \frac{dy}{y^2} = xdx \\
                       y \ne 0
                   \end{array}\right.
        \end{array}\right. \iff \int \frac{dy}{y^2} = \int xdx \implies -\frac{1}{y} = \frac{x^2}{2} + C$ \\

    $\left[\begin{array}{l}
            y = -\frac{2}{x^2 + 2C}, \ C \in \mathbb{R} \\
            y = 0
        \end{array}\right.$
\end{example}

\begin{theorem}[Задача Коши]
    $\left\{\begin{array}{l}
            y'=f(x,y) \\
            y(x_0) = y_0
        \end{array}\right.$ \\

    $f(x,y) \in C(D), \quad (x_0, y_0) \in D$ (РИСУНКИ)
\end{theorem}

\begin{example}
    $y' = \sqrt{y}$ \\

    $\left[\begin{array}{l}
            y = 0 \\
            \left\{\begin{array}{l}
                       \frac{dy}{\sqrt{y}} = \int dx \\
                       y \ne 0
                   \end{array}\right.
        \end{array}\right. \iff 2\sqrt{y} = x + C \implies y = (\frac{x + c}{2})^2$ при $x + c \geqslant 0$. \\

    \begin{enumerate}
        \item $y = 0 \ \cup$ парабола $AB_1D_1$;
        \item $x_0$ на кривой $y = 0 \left[\begin{array}{l}
                      y = 0   \\
                      ABD     \\
                      AB_1D_1 \\
                      AB_2D_2
                  \end{array}\right.$
    \end{enumerate}

    Ответ: $\left[\begin{array}{l}
            y = 0 \\
            y = (\frac{x + c}{2})^2, \quad x + c \geqslant 0
        \end{array}\right.$
\end{example}

\begin{definition}[Точка единственности решения]
    Точка $(x_0, y_0)$ называется \textbf{точкой единственности решения} $y = \phi(x)$, если через нее не проходит другое решение, не совпадающее с решением $y = \phi(x)$ ни в какой окрестности этой точки.

    Остальные точки называются \textbf{точками неединственности}.

    Решение, которое содержит точки неединственности, называется \textbf{особым решением}.
\end{definition}

\begin{theorem}[$\exists$ и $!$-ть решения задачи Коши]
    Пусть
    \begin{center}
        $f(x,y)$ в $\left\{\begin{array}{l}
                y'=f(x,y) \\
                y(x_0) = y_0
            \end{array}\right.$
    \end{center}

    \begin{enumerate}
        \item Определена и непрерывна в прямоугольнике в прямоугольнике \\
              $\Pi = \{(x,y): \ |x - x_0| \leqslant a, \ |y - y_0| \leqslant b\}$
        \item Удовлетворяет условию Липшица по $y$ в $\Pi$ ($f_y'(x,y)$ непрерывна в $\Pi$)
    \end{enumerate}

    Тогда $\exists !$ решение задачи $\left\{\begin{array}{l}
            y'=f(x,y) \\
            y(x_0) = y_0
        \end{array}\right.$ в окрестности точки \\
    $x_0 \ (x_0 - h, \ x_0 + h)$, где $h = \min(a, \frac{b}{M}), \ M = \max|f(x,y)|, \ (x,y) \in \Pi$. (РИСУНОК)
\end{theorem}

\begin{definition}
    $f(x,y)$ удовлетворяет условию Липшица по переменной $y$, если $\exists L > 0$ такая, что $\forall (x,y_1)$ и $(x,y_2)$ имеет место $|f(x,y_1) - f(x,y_2)| \leqslant L \cdot|y_1 - y_2|$.

    Если $f_y'(x,y)$ - непрерывна в $\Pi$, то выполняется условие Липшица.

    $\forall (x,y_1), \ (x,y_2) \in \Pi, \ \exists \widetilde{y} \in [y_1,y_2]$.

    $|f(x,y_1) - f(x,y_2)| \leqslant |f_y'(x,\widetilde{y}) \cdot (y_1 - y_2)| \leqslant |f_y'(x,\widetilde{y})||y_1 - y_2| = L|y_1 - y_2|$.
\end{definition}

\begin{example}
    $y' = \frac{1}{y^2}, \ f(x,y) = \frac{1}{y^2}, \ f_y'=\frac{2}{y^3}, \quad \int y^2dy = \int ydx \implies \frac{y^3}{3} = x + c \implies
        \left\{\begin{array}{l}
            y = \sqrt[3]{3(x + c)} \\
            y(x_0) = y_0
        \end{array}\right. \implies y = \sqrt[3]{3(x-x_0) + y_0^3}$
\end{example}

\begin{example}
    $y'=sign x = \left\{\begin{array}{rl}
            1,  & x > 0 \\
            0,  & x = 0 \\
            -1, & x < 0
        \end{array}\right.$
\end{example}

\begin{example}
    $y' = y^2 - 2y + 1 = (y-1)^2$

    $\left[\begin{array}{l}
            y = 1 \\
            \left\{\begin{array}{l}
                       y \ne 1 \\
                       \frac{dy}{(y - 1)^2} = \int dx
                   \end{array}\right.
        \end{array}\right. \iff \frac{1}{y-1} = x + C \implies y = 1 - \frac{1}{x + C}$
\end{example}

\begin{center}
    {\Large ПОСЛЕ ЭТОГО ИДЕТ ТО, ЧТО Я ПРОПУСТИЛ}
\end{center}

\section{Уравнение Бернулли}

\begin{equation*}
    y'+p(x) \cdot y = q(x)\cdot y^m, \quad m\ne 1
\end{equation*}

\begin{center}
    $y=0$ -- решение при $m > 0$
\end{center}

\begin{enumerate}
    \item Сведение к линейному
    \item Метод Бернулли
\end{enumerate}

\begin{enumerate}
    \item $y^m \ne 0, \quad \frac{y'}{y^m} + P(x) \cdot \frac{y}{y^m} = q(x); \quad z = y^{1-m}, \quad z' = (1-m)\cdot y^{-m}\cdot y' = (1-m)\cdot \frac{y'}{y^m}$ \\

          $\frac{z'}{1-m} + P(x) \cdot z = q(x) \ \big| \ \cdot (1-m)$ \\

          $z' + (1-m) \cdot P(x) \cdot z = (1-m)\cdot q(x)$

    \item Пусть $y = u\cdot v, \quad u' \cdot v + u \cdot v' + P(x) \cdot u \cdot v = q(x) \cdot u^m \cdot v^m$ \\

          $u' \cdot v  + u(v' + P(x) \cdot v) = q(x) \cdot u^m\cdot v^m$ \\

          $\left\{\begin{array}{l}
                  v' + P(x) \cdot v = 0 \\
                  u' = q(x) \cdot u^m \cdot v^{m-1}
              \end{array}\right. \implies u = e^{-\int p(x)dx}$ \\

          $\frac{u'}{u^m} = q(x)(e^{-\int p(x)dx})^{m-1} \implies u \implies y = u \cdot v$
\end{enumerate}

\section{Уравнения в полных дифференциалах}

\begin{definition}[Уравнение в ПД]
    Уравнение вида
    \begin{equation}
        P(x,y)dx + Q(x,y)dy = 0
    \end{equation}
    называется \textbf{уравнением в полных дифференциалах (ПД)}, если левая часть уравнения $(1.5)$ является дифференциалом накоторой функции.
    \begin{equation}
        P,Q,P_x,Q_x,P_y,Q_y \in C(D),
    \end{equation}
    \begin{center}
        $D$ - односвязная область в $\mathbb{R}^2$
    \end{center}
\end{definition}

\begin{theorem}
    Если существует такая функция $y(x,y): \ du = Pdx + Qdy$, выполняются условия $(1.6)$, то имеет место в $D$
    \begin{equation}
        \frac{\delta Q}{\delta x} = \frac{\delta P}{\delta y}
    \end{equation}
\end{theorem}

\begin{proof}
    Пусть $\exists u(x,y): \ du = Pdx + Qdy, \quad du = \frac{\delta u}{\delta x}dx + \frac{\delta u}{\delta y}dy,\\ U\in C^2(x)$ \\

    $\left\{\begin{array}{l}
            \frac{\delta u}{\delta x} = P \\
            \frac{\delta u}{\delta y} = Q
        \end{array}\right. \iff \frac{\delta^2 u}{\delta y \delta x} = \frac{\delta P}{\delta y} = \frac{\delta^2 u}{\delta x \delta y} = \frac{\delta Q}{\delta x} \iff \frac{\delta P}{\delta y} = \frac{\delta Q}{\delta x}$.
\end{proof}

\begin{theorem}
    Для $\exists$ функции $u(x,y)$ такой, что $du = Pdx + Qdy$ при выполнении $(1.6) \iff \frac{\delta Q}{\delta x} = \frac{\delta P}{\delta y}$.
    \begin{enumerate}
        \item $Pdx + Qdy = 0$
        \item $du = Pdx + Qdy \implies du = 0$
        \item $u(x,y) = C$
    \end{enumerate}

    \textbf{Общий интеграл} - это функция $u(x,y)$, которая равна константе на решении уравнения.
\end{theorem}

\begin{proof}
    Восстановление функции $u(x,y)$ по ее полному дифференциалу.

    Пусть выполняется $(1.6), \ D$ -- односвязная область в $\mathbb{R}^2, \quad du = Pdx + Qdy$.

    Задача: найти $u(x,y) \in C^2(D)$ \\

    $du = \frac{\delta u}{\delta x}dx + \frac{\delta u}{\delta y}dy, \left\{\begin{array}{l}
            \frac{\delta u}{\delta x} = P(x,y) \\
            \frac{\delta u}{\delta y} = Q(x,y)
        \end{array}\right.$

    Проинтегрируем $1$-е уравнение: $(x_0,y_0) \in D$
    \begin{equation*}
        \int_{x_0}^{x}\frac{\delta u}{\delta x}dx = \int_{x_0}^{x}P(x,y)dx
    \end{equation*}
    \begin{equation}
        u(x,y) = u(x_0,y_0) + \int_{x_0}^{x}P(x,y)dx
    \end{equation}

    $\frac{\delta u}{\delta y} = \frac{\delta}{\delta y}(u(x_0,y) + \int_{x_0}^{x}P(x,y)dx) = \frac{\delta u(x_0,y)}{\delta y} + \int_{x_0}^{x}\frac{\delta P(x,y)}{\delta y}dx = \frac{\delta u(x_0,y)}{\delta y} + \int_{x_0}^{x}\frac{\delta Q}{\delta x}dx = \frac{\delta u(x_0,y)}{\delta y} + Q(x,y) - Q(x_0,y) = Q(x,y) \implies \frac{\delta u(x_0,y)}{\delta y} = Q(x_0,y)$, интегрируем по $y$:

    $\int_{y_0}^{y}\frac{\delta u(x_0,y)}{\delta y}dy = \int_{y_0}^{y}Q(x_0,y)dy, \quad u(x_0,y) - u(x_0,y_0) = \int_{y_0}^{y}Q(x_0,y)dy$
    \begin{equation}
        u(x,y) = u(x_0,y_0) + \int_{x_0}^{x}P(x,y)dx + \int_{y_0}^{y}Q(x_0,y)dy
    \end{equation}
    \begin{equation}
        u(x,y) = u(x_0,y_0) + \int_{x_0}^{x}P(x,y_0)dx + \int_{y_0}^{y}Q(x,y)dy
    \end{equation}
    \begin{center}
        $du(x,y) = 0, \quad u(x,y) = C$
    \end{center}
\end{proof}

\begin{example}
    $ydx + xdy = 0 \iff \frac{dx}{x} = - \frac{dy}{y}, \quad y = P, \ x = Q$

    \begin{enumerate}
        \item $\frac{\delta Q}{\delta x} = \frac{\delta P}{\delta y} = 1 \implies$ уравнение в ПД.

              \begin{equation*}
                  d(x\cdot y) = dx \cdot y + xdy = 0, \quad x \cdot y = C
              \end{equation*}

        \item $\left\{\begin{array}{l}
                      \frac{\delta u}{\delta x} = y = P \implies \int\frac{\delta u}{\delta x}dx = \int ydx \implies u(x,y) = y\cdot x + C(y) \\
                      \frac{\delta u}{\delta y} = x = Q \implies \frac{\delta u}{\delta y} = \frac{\delta}{\delta y}(y\cdot x + C(y)) = x + C'(y) = x
                  \end{array}\right.$

              \begin{equation*}
                  y(x,y) = y\cdot x + C \implies y \cdot x + C = C_1 \implies y \cdot x = \widetilde{C}, \ \widetilde{C} = C_1 - C
              \end{equation*}

              $\left\{\begin{array}{l}
                      \frac{\delta u}{\delta x} = y \\
                      \frac{\delta u}{\delta y} = x
                  \end{array}\right\} \implies \int \frac{\delta u}{\delta y}dy = \int x \cdot dy \implies$
              \begin{equation*}
                  u(x,y) = x\cdot y + C(x)
              \end{equation*}

              $\frac{\delta u}{\delta x} = \frac{\delta}{\delta x}(x\cdot y) + C(x) = y + C'(x) = y \implies C'(x) = 0 \implies C(x) = C_1 \implies u(x,y) = x\cdot y + C_1;$
              \begin{equation*}
                  u(x,y) = C_2 \implies x\cdot y + C_1 = C_2 \implies x\cdot y = C, \quad C = C_2 - C_1
              \end{equation*}
    \end{enumerate}
\end{example}

\section{Интегральный множитель}

$\frac{1}{y^2}: \ ydx - xdy = 0$

$\frac{\delta P}{\delta y} = y_y' = 1 \ne \frac{\delta Q}{\delta x} = \frac{\delta}{\delta x}(-x) = -1$
\begin{equation*}
    u(x,y) = C
\end{equation*}

\begin{equation*}
    \frac{ydx - xdy}{y^2} = 0 \implies d(\frac{x}{y}) = 0
\end{equation*}

\begin{equation*}
    \frac{x}{y} = C, \quad y = 0
\end{equation*}

\begin{definition}
    Пусть
    \begin{equation}
        M(x,y)dx + N(x,y)dy = 0
    \end{equation}
    не является уравнением в ПД, $M,N \in C^2(D), \ D$ -- односвязная область в $\mathbb{R}^2$.

    $\mu(x,y)$ называется \textbf{интегрирующим множителем} уравнения $(1.11)$, если $\mu(x,y)M(x,y)dx + \mu (x,y)N(x,y)dy$ -- является ПД некоторой функции
    \begin{equation*}
        \frac{\delta P}{\delta y} = \frac{\delta Q}{\delta x}
    \end{equation*}
    \begin{center}
        {\Large НАДО ДОПИСАТЬ}
    \end{center}
\end{definition}

\section{Методы построения интегрирующего множителя}

\begin{equation}
    M(x,y)dx + N(x,y)dy = 0
\end{equation}
\begin{equation*}
    M,N \in C^2(D), \quad \frac{\delta N}{\delta x} \ne \frac{\delta M}{\delta y}, \quad \mu(x,y)\in C^1(D)
\end{equation*}
\begin{equation*}
    \mu(x,y)\cdot M(x,y)dx + \mu(x,y) \cdot N(x,y)dy = 0
\end{equation*}
\begin{equation*}
    \frac{\delta Q}{\delta x} = \frac{\delta P}{\delta y}
\end{equation*}
\begin{multline*}
    \frac{\delta \mu(x,y)}{\delta x} \cdot N(x,y) + \mu(x,y)\cdot \frac{\delta N(x,y)}{\delta x} = \\
    = \frac{\delta \mu(x,y)}{\delta y} \cdot M(x,y) + \mu(x,y) \cdot \frac{\delta M(x,y)}{\delta y}
\end{multline*}
\begin{enumerate}
    \item \begin{multline*}
              \mu = \mu(x) \implies \frac{\delta \mu}{\delta y} = 0 \implies \\
              \implies \mu'(x) \cdot N(x,y) + \mu(x)\cdot \frac{\delta N}{\delta x} = \mu(x)\cdot \frac{\delta M}{\delta y}
          \end{multline*}

          \begin{equation*}
              \frac{\mu'(x)}{\mu(x)} = \frac{\frac{\delta M}{\delta y} - \frac{\delta N}{\delta x}}{N} = F(x)
          \end{equation*}

          \begin{equation*}
              \int\frac{\mu'(x)}{\mu(x)}dx = \int F(x)dx
          \end{equation*}

          \begin{equation*}
              \ln|\mu(x)| = \ln c + \int F(x)dx, \quad \mu(x) = c\cdot e^{\int F(x)dx} \underset{c = 1}{=} e^{\int F(x)dx}
          \end{equation*}

    \item \begin{multline*}
              \mu = \mu(y) \implies \frac{\delta \mu}{\delta x} = 0 \implies \\
              \implies \mu'(y) \cdot M(x,y) + \mu(y)\cdot \frac{\delta M}{\delta y} = \mu(y)\cdot \frac{\delta N}{\delta x}
          \end{multline*}

          \begin{equation*}
              \frac{\mu'(y)}{\mu(y)} = \frac{\frac{\delta N}{\delta x} - \frac{\delta M}{\delta y}}{M} = F(y)
          \end{equation*}

          \begin{equation*}
              \int\frac{\mu'(y)}{\mu(y)}dy = \int F(y)dy
          \end{equation*}

          \begin{equation*}
              \ln|\mu(y)| = \ln c + \int F(y)dy, \quad \mu(y) = c\cdot e^{\int F(y)dy} \underset{c = 1}{=} e^{\int F(y)dy}
          \end{equation*}

    \item \begin{equation*}
              \mu = \mu(\omega(x,y))
          \end{equation*}

          \begin{eqnarray*}
              \frac{\delta \mu}{\delta \omega} \cdot \frac{\delta \omega}{\delta x} \cdot N + \mu \cdot \frac{\delta N}{\delta x} & = &\\
              & = & \frac{\delta \mu}{\delta \omega} \cdot \frac{\delta \omega}{\delta y} \cdot M + \mu \cdot \frac{\delta M}{\delta y}
          \end{eqnarray*}

          \begin{eqnarray*}
              \frac{\frac{\delta \mu}{\delta \omega}}{\mu(\omega)} = \frac{\frac{\delta M}{\delta y} - \frac{\delta N}{\delta x}}{N\cdot \frac{\delta \omega}{\delta x} - M\cdot \frac{\delta \omega}{\delta y}} = F(\omega) & \implies & \\
              & \implies & \mu(\omega) = e^{F(\omega)d\omega}
          \end{eqnarray*}
\end{enumerate}

\begin{example}
    \begin{equation*}
        (x^2 + y^2 + x)dx + ydy = 0, \quad M = x^2 + y^2 + x, \quad N = y
    \end{equation*}
    \begin{eqnarray*}
        \frac{\delta N}{\delta x} = 0 & \ne & \\
        & \ne & 2y = \frac{\delta M}{\delta y}
    \end{eqnarray*}
    \begin{multline*}
        \mu = \mu(x) = ?, \quad \mu(x^2 + y^2 + x)dx + \mu y dy = 0, \\
        P = \mu(x^2 + y^2 + x), \quad Q = \mu y, \\
        \frac{\delta P}{\delta y} = \frac{\delta Q}{\delta x}, \quad \frac{\delta M}{\delta y}(x^2 + y^2 + x) + \mu \cdot 2y = \frac{\delta \mu}{\delta x} \cdot y + \mu \cdot 0, \\
        \frac{\mu'(x)}{M} = \frac{2y}{y} = 2, \quad \mu(x) = e^{2x}, \\
        e^{2x}(x^2 + y^2 + x) dx + e^{2x} \cdot y dy = 0, \\
        P = e^{2x}(x^2 + y^2 + x), \quad Q = e^{2x}\cdot y, \\
        \frac{\delta P}{\delta y} = 2e^{2x} \cdot y = \frac{\delta Q}{\delta x}, \\
        \left\{\begin{array}{l}
            \frac{\delta u}{\delta x} = e^{2x}(x^2 + y^2 + x) \\
            \frac{\delta u}{\delta y} = e^{2x}\cdot y
        \end{array}\right. \implies u(x,y) = e^{2x} \cdot \frac{y^2}{2} + c(x), \\
        u_x' = 2e^{2x} \cdot \frac{y^2}{2} + c'(x) = e^{2x}(x^2 + y^2 + x), \quad c'(x) = e^{2x}(x^2 + x)
    \end{multline*}
    \begin{eqnarray*}
        c(x) = \frac{e^{2x}}{2}(x^2 + x) - \int \frac{e^{2x}}{2}(2x + 1)dx & = & \\
        = \frac{e^{2x}}{2}(x^2 + x) - \frac{e^{2x}}{4}(2x + 1) &+& \int \frac{e^{2x}}{4}\cdot 2 dx = \\
        = \frac{e^{2x}}{2}(x^2 + x) &-& \frac{e^{2x}}{4}(2x+1) + \frac{e^{2x}}{4} + C
    \end{eqnarray*}
    \begin{equation*}
        c(x) = \frac{e^{2x}}{2}\cdot x^2 + C
    \end{equation*}
    \begin{equation*}
        u(x,y) = e^{2x} \cdot \frac{x^2 + y^2}{2} + C = \widetilde{C}
    \end{equation*}
    \begin{center}
        $e^{2x} \cdot \frac{x^2 + y^2}{2} = C$ -- общий интеграл
    \end{center}
    \begin{enumerate}
        \item Если $\mu_0$ -- ИМ, то $\forall c \in \mathbb{R} \quad \mu_1 = C \cdot \mu_0$ -- тоже является ИМ
        \item Пусть $\mu_0$ -- ИМ уравнения $(1.12)$, $V_0$ -- соответствующий ему интеграл, то есть
              \begin{equation*}
                  \mu_0 \cdot Mdx + \mu_0 \cdot N dx = d V_0,
              \end{equation*}
              тогда для произвольной функции $\phi \in C^1(D), \ \phi \ne 0$, $\mu_1 = \mu_0 \cdot \phi(V_0)$ -- так же является ИМ.
              \begin{multline*}
                  Mdx + Ndy = 0, \quad \mu_1 \cdot Mdx + \mu_1 \cdot N dy = \\
                  = \mu_0 \cdot \phi(V_0)\cdot Mdx + \mu_0 \cdot \phi(V_0)\cdot Ndy = \\
                  = \phi(V_0)(\mu_0 \cdot Mdx + \mu_0 \cdot N dy) = \phi(V_0)dV_0 = \\
                  = d\bigg(\int \phi(V_0)dV_0\bigg) = dV_1, \quad \int \phi(V_0)dV_0 = V_1
              \end{multline*}
        \item Если $\mu_1$ и $\mu_2$ -- интегральные множители уравнения $(1.12)$, тогда
              \begin{equation*}
                  \mu_2 = \mu_1 \cdot \phi(V_1),
              \end{equation*}
              где $\phi$ -- произвольная функция класса $C^1$, $V_1$ -- соответствующий интеграл для $\mu_1$.
    \end{enumerate}
\end{example}

\begin{effect}
    Если $\mu_1$ и $\mu_2$ -- интегральные множители уравнения $(1.12)$ и $\frac{\mu_1}{\mu_2} \ne const$, тогда $\frac{\mu_1}{\mu_2}$ -- является интегралом для уравнения $(1.12)$.
\end{effect}

\begin{theorem}
    Если уравнение 1-го порядка имеет общий интеграл $u(x,y) = C$, то оно имеет интегрирующий множитель.
\end{theorem}

\begin{proof}
    \begin{equation*}
        u(x,y) = C\left\{\begin{array}{l}
            Mdx + Ndy = 0 \\
            du \equiv \frac{\delta u}{\delta x}dx + \frac{\delta u}{\delta y}dy = 0
        \end{array}\right.
    \end{equation*}
    $(dx,dy)$ -- ненулевое решение если определитель равен 0, то есть
    \begin{equation*}
        \left|\begin{array}{cc}
            M                         & N                         \\
            \frac{\delta u}{\delta x} & \frac{\delta u}{\delta y}
        \end{array}\right| = M\cdot \frac{\delta u}{\delta y} - N\cdot \frac{\delta u}{\delta x} = 0
    \end{equation*}
    \begin{multline*}
        M\cdot \frac{\delta u}{\delta y} = N\cdot \frac{\delta u}{\delta x} \quad \bigg| \ \cdot (MN) \implies \frac{1}{N}\cdot \frac{\delta u}{\delta y} = \frac{1}{M} \cdot \frac{\delta u}{\delta x} \overset{?}{=} \mu, \\
        \mu \cdot Mdx + \mu \cdot Ndy = 0, \quad \frac{1}{M} \frac{\delta u}{\delta x} \cdot Mdx + \frac{1}{N} \cdot \frac{\delta u}{\delta y} \cdot Ndy = 0, \\
        \frac{\delta u}{\delta x}dx + \frac{\delta u}{\delta y}dy = 0 \implies du = 0
    \end{multline*}
\end{proof}

\section*{Еще один способ построения интегрального множителя}
\begin{equation*}
    \underset{\RomanNumeralCaps{1}}{M_1dx + N_1dy} + \underset{\RomanNumeralCaps{2}}{M_2dx + N_2dy} = 0
\end{equation*}

Пусть $\mu_1$ -- интегральный множитель для $\RomanNumeralCaps{1}$, $V_1$ -- соответствующий ему интеграл, то есть
\begin{equation*}
    dV_1 = \mu_1 \cdot M_1 dx + \mu_2 \cdot N_2 dy,
\end{equation*}
$\mu_2$ -- интегральный множитель для $\RomanNumeralCaps{2}$, $V_2$ -- соответствующий ему интеграл, то есть
\begin{equation*}
    dV_2 = \mu_2 \cdot M_2 dx + \mu_2 \cdot N_2 dy,
\end{equation*}
тогда $\exists \phi,\psi \in C^1(D): \quad \mu_1 \cdot \phi(V_1) = \mu_2 \cdot \psi(V_2)$ и $\mu = \mu_1 \cdot \phi (V_1)$ или $\mu = \mu_2 \cdot \psi(V_2)$ -- будет интегральным множителем.

\begin{example}
    \begin{equation*}
        (\frac{y}{x} + 3x^2)dx + (1 + \frac{x^3}{y})dy = 0
    \end{equation*}
    \begin{equation*}
        (\frac{y}{x} + dy) + (3x^2 dx + \frac{x^3}{y}dy) = 0
    \end{equation*}
    \begin{minipage}{0.4\textwidth}
        $\frac{y}{x} + dy = 0$ \\
        \begin{equation*}
            \mu_1 = x
        \end{equation*}
        $ydx + xdy = 0$ \\
        $d(xy) = 0 \implies xy = C_1$ \\
        \begin{equation*}
            V_1 = xy
        \end{equation*}
    \end{minipage}
    \hfill
    \begin{minipage}{0.4\textwidth}
        $\frac{y}{x} + dy = 0$ \\
        \begin{equation*}
            \mu_1 = x
        \end{equation*}
        $ydx + xdy = 0$ \\
        $d(xy) = 0 \implies xy = C_1$ \\
        \begin{equation*}
            V_1 = xy
        \end{equation*}
    \end{minipage}
\end{example}

\begin{center}
    {\Large ДОПИСАТЬ НАДО}
\end{center}

\end{document}